\title{Probabilistic models}
\author{Vicente González Ruiz}
\maketitle
\tableofcontents

\section{Funtionality}
\begin{itemize}
\item Compute the probabilities of the symbols (see
  Figure~\ref{fig:compresion_entropica}).
\item Both, the encoder and the decoder must be syncronized.
\end{itemize}

\section{Static models}
\begin{itemize}
\item Static models are the simplest ones because they suppose that
  the probabilities of the symbols remain constant through the
  compression/decompression.
\item In this case, variable-length code-words can be precomputed.
\item Static models are very common in codecs such
  as JPEG and MPEG (audio and video).
\end{itemize}

\section*{Let's go the the lab!}
\begin{enumerate}
\item Download
  \url{http://www.ace.ual.es/~vruiz/doctorado/text-compression.tar.gz}
  and \texttt{make huff\_s0}.
\item Encode each image of
  \href{http://www.ace.ual.es/~vruiz/images/}{The Test Image
    Corpus} using the  codec. Continue filling the table:
\begin{verbatim}
    Codec | lena boats pepers zelda
----------+------------------------
      RLE | ....  ....   ....  ....
  BWT+RLE | ....  ....   ....  ....
     LZSS | ....  ....   ....  ....
      LZW | ....  ....   ....  ....
  huff_s0 | ....  ....   ....  ....
\end{verbatim}
\item Check that the \texttt{huff\_s0} codec is lossless.
\end{enumerate}

\section{Adaptive models}
\begin{itemize}
\item The probabilities of the symbols are computed while they arrive to the compressor.
\item In general, the compression ratios that adaptive models reach are better than static model's ones, because the
  probabilities of the symbols are localy computed (think of the sequence `aaaaaaaaaaaaaabbbbbbbbbbbbbbb`).
\end{itemize}

\section{Encoding}
\begin{enumerate}
\item Asign the same probability to all the symbols of the source alphabet.
\item While the input if not exhausted:
  \begin{enumerate}
  \item Encode the next symbol.
  \item Update (increase) its probability.
  \end{enumerate}
\end{enumerate}

\section{Decoding}
\begin{enumerate}
\item Identical to the step 1 of the encoder.
\item While the input is not exhausted:
  \begin{enumerate}
  \item Decode the next symbol.
  \item Identical to the step 2.b of the encoder.
  \end{enumerate}
\end{enumerate}

\section*{Let's go the the lab!}
\begin{enumerate}
\item Download
  \url{http://www.ace.ual.es/~vruiz/doctorado/progs.tar.gz} and
  \texttt{make huff\_a0} and \texttt{make arith\_a0}.
\item Encode each image of
  \href{http://www.ace.ual.es/~vruiz/images/}{The Test Image
    Corpus} using the  codec. Continue filling the table:
\begin{verbatim}
    Codec | lena boats pepers zelda Average
----------+--------------------------------
        : |    :     :      :     :       :
  huff_s0 | ....  ....   ....  ....    ....
  huff_a0 | ....  ....   ....  ....    ....
 arith_a0 | ....  ....   ....  ....    ....
\end{verbatim}
\item Remember to check that both codecs are lossless.
\end{enumerate}

\section{Initially empty models}
\label{modelo_probabilistico_adaptativo_vacio}
\begin{itemize}
\item The smaller the number of symbols used by the model, the higher
  the probabilities, and therefore, the better the compression ratios.
\item An initially empty model only stores the $\mathtt{ESC}$(cape) symbol (a
  symbol that it is used by the encoder only when a new symbol is
  found and therefore, it must be part of the encoder's and decoder's models).
\item We assume that arithmetic coding is used and therefore, when we
  input or output $c(s)$, we are transmitting $I(s)$ bits of code.
\end{itemize}

\section{Encoder}
\begin{enumerate}
\item Set the probability of the $\mathtt{ESC}$ symbol to $1.0$ (and the probability of
  the rest of the symbols to $0.0$).
\item While the input is not exhausted:
  \begin{enumerate}
  \item $s\leftarrow$ next symbol.
  \item If $s$ has been found before, then:
    \begin{enumerate}
    \item Output $c(s)$ (encode).
    \end{enumerate}
  \item Else:
    \begin{enumerate}
    \item Output $c(\mathtt{ESC})$.
    \item Output a raw symbol $s$.
    \end{enumerate}
  \item Update $p(s)$.
  \end{enumerate}
\end{enumerate}

\section{Decoder}
\begin{enumerate}
\item Identical to the step 1 of the encoder.
\item While the input is not exhausted:
  \begin{enumerate}
  \item $c(s)\leftarrow $ next code-word.
  \item Decode $s$.
  \item If $s=\mathtt{ESC}$, then:
    \begin{enumerate}
    \item Input a raw symbol $s$.
    \end{enumerate}
  \item Update $p(s)$.
  \item Output $s$.
  \end{enumerate}
\end{enumerate}

\section*{Let's go the the lab!}
\begin{enumerate}
\item \texttt{make arith-n-c} and \texttt{make arith-n-d}.
\item
\begin{verbatim}
         Codec | lena boats pepers zelda Average
---------------+--------------------------------
             : |    :     :      :     :       :
arith-n-c -o 0 | ....  ....   ....  ....    .... #1
 BWT | ahuff_0 | ....  ....   ....  ....    ....
      bzip2 -9 | ....  ....   ....  ....    .... #2
       gzip -9 | ....  ....   ....  ....    .... #3

#1 -> Similar to arith_a0 but using an initially
      empty model.

#2 -> Similar to BWT | ahuff_0

#3 -> Similar to LZ77 + ahuff_0
\end{verbatim}
\item Check the reversibility.
\end{enumerate}

\section{Models with memory}
\begin{itemize}
\item In most cases, the probability of a symbol depends on its
  neighborhood (context).
\item The higher the memory of the model (the context), the higher the
  accuracy of the predictions (probabilities), and therefore, the
  lower the length of the code-words \cite{cleary1984data}.
\item Let ${\cal C}[i]$ the last $i$ encoded symbols and $p(s|{\cal C}[i])$ the
  probability that the symbol $s$ follows the context ${\cal C}[i]$.
\item Let $k$ the maximal order of the prediction (i.e. the largest
  number of symbols of ${\cal C}[]$ that are going to be used as the
  actual context). Notice that ${\cal C}[0]=\varnothing$ and the model
  has no memory.
\item We suppose that arithmetic coding is used and therefore, when we
  input or output $c(s)$, we are transmitting $I(s)$ bits of code.
\item Let $r$ the size of the source alphabet.
\end{itemize}

\section{Encoder}
\begin{enumerate}
\item Create an empty model for every possible context of length $0\le i \le k$.
\item Create an non-empty model for $k=-1$.
\item While the input is not exhausted:
  \begin{enumerate}
  \item $s\leftarrow$ Input ${\log_2(r)}$ bits. \# Input a symbol.
  \item $i\leftarrow k$ (except for the first symbol, where
    $i\leftarrow 0$).
  \item $i\leftarrow k$ (except for the first symbol, where
    $i\leftarrow 0$):
    \begin{enumerate}
    \item Output $\leftarrow c(\text{ESC}|{\cal C}[i])$.
    \item Update $p(\text{ESC}|{\cal C}[i])$.
    \item Update $p(s|{\cal C}[i])$ (insert $s$ into the ${\cal C}[i]$ context).
    \item $i\leftarrow i-1$.
    \end{enumerate}
  \item Output $\leftarrow c(s|{\cal C}[i])$. The symbols that were found in the models for
       contexts with order $>i$ must be excluded of the actual (${\cal C}[i]$) context-model because, for sure, $s$ is not any of them.
  \item If $i\ge 0$, update $p(s|{\cal C}[i])$.
  \end{enumerate}
\end{enumerate}

\section*{Example}
\begin{itemize}
\item Let $r=256$ the size of the source alphabet.
\item The probabilistic model $M[{\cal C}[-1]]$ (for the special context
  ${\cal C}[-1]$) is static, non empty and has an special symbol $\mathtt{EOF}$
  (End Of File) that is going to be used when the compression has
  finished:
  $$M[{\cal C}[-1]]=\{0,1~1,1~\cdots~\mathtt{a},1~\mathtt{b},1~\cdots~255,1~\mathtt{EOF},1\}=A.$$
  In a pair $(a,b)$, $a$ is the symbol and $b$ is its probability (represented by a counter of ocurrences of $s$ in the processed input until that moment).

\item $M[{\cal C}[0]]$ is adaptative and empty:
  $$M[{\cal C}[0]]=\{\mathtt{ESC},1\}.$$

\item In this example (for the sake of the simplicity), the maximal
  order of the prediction $k=1$ (we only remember the previous
  symbol). Therefore, there are $r=256$ probabilistic models:
  $$M[{\cal C}[1]]=\{\mathtt{ESC},1\}, 0\le {\cal C}[1]\le r.$$

  \begin{enumerate}
  \item [3.a] $s\leftarrow \mathtt{a}$.
  \item [3.b] $i\leftarrow 0$ (we don't know the previous symbol).
  \item [3.c] $p(\mathtt{a}|{\cal C}[0])=0$ (the context $M[C[0]]$ has only the $\mathtt{ESC}$ symbol, $M[C[0]]=\{\mathtt{ESC},1\}$).
  \item [3.c.i] Output $\leftarrow c(\mathtt{ESC}|{\cal C}[0])$ (in fact, zero bits because 
      $l(c(\mathtt{ESC}|{\cal C}[0]))=0$).
  \item [3.c.ii] Update $p(\mathtt{ESC}|{\cal C}[0])$ (now, $M[C[0]]=\{\mathtt{ESC},2\}$).
  \item [3.c.iii] [3.C.c] Insert symbol $\mathtt{a}$ into
    $M[{\cal C}[0]]=\{\mathtt{ESC},2~\mathtt{a},1\}$.
  \item [3.c.iv] \item [3.c.iv]
  \item [3.c]  $p(\mathtt{a}|{\cal C}[-1])\neq 0$.
  \item [3.d] Output $\leftarrow c(\mathtt{a}|{\cal C}[-1])$ where
    $p(\mathtt{a}|{\cal C}[-1]) = 1/(256+1)$.
  \end{enumerate}
  \item Encoding of the second symbol (\texttt{b}):
  \begin{enumerate}
  \item [3.a] $s\leftarrow \mathtt{b}$.
  \item [3.b] $i\leftarrow 1$.
  \item [3.c] $p(\mathtt{b}|{\cal C}[1])=0$ because ${\cal C}[1]=\mathtt{a}$ and $M[\mathtt{a}]=\{\mathtt{ESC},1\}$.
  \item [3.c.i] Output $\leftarrow c(\mathtt{ESC}|\mathtt{a})$ with $l(c(\mathtt{ESC}|\mathtt{a}))=0$.
  \item [3.c.ii] Update $p(\mathtt{ESC})$ in $M[\mathtt{a}]$ (now, $M[a]=\{\mathtt{ESC},2\}$).
  \item [3.c.iii] Insert $\mathtt{b}$ into $M[\mathtt{a}]=\{\mathtt{ESC},2~ \mathtt{b},1\}$.
  \item [3.c.iv] $i\leftarrow 0$.
  \item [3.c] $p(\mathtt{b}|{\cal C}[0])=0$ $M[{\cal C}[0]]=\{\mathtt{ESC},2~\mathtt{a},1\}$.
  \item [3.c] $p(\mathtt{b}|{\cal C}[0])=0$ because $M[{\cal C}[0]]=\{\mathtt{ESC},2~\mathtt{a},1\}$.
  \item [3.c.i] Output $\leftarrow c(\mathtt{ESC}|{\cal C}[0])$, where $p(\mathtt{ESC}|{\cal C}[0]) = 2/3$.
  \item [3.c.ii] Update $p(\mathtt{ESC}|{\cal C}[0])$ (now, the count of $\mathtt{ESC}$ is $3$).
  \item [3.c.iii] Insert $\mathtt{b}$ into $M[{\cal C}[0]] = \{\mathtt{ESC},3~\mathtt{a},1~\mathtt{b},1\}$.
  \item [3.c.iv] $i\leftarrow -1$.
  \item [3.c] $p(\mathtt{b}|{\cal C}[-1])\neq 0$.
  \item [3.d] Output $\leftarrow c(\mathtt{b}|{\cal C}[-1])$ where $p(\mathtt{b}|{\cal C}[-1]) = 1/r$. The symbol $\mathtt{a}$ has been excluded of the calculus of the probability of $\mathtt{b}$ for context $C[-1]$ because $\mathtt{a}\in M[{\cal C}[0]] = \{\mathtt{ESC},3~ \mathtt{a},1~ \mathtt{b},1\}$.
      \end{enumerate}
\end{itemize}

\begin{comment}
\begin{figure}
\begin{center}
  \begin{tabular}{crrl}
    Input & Output & Prob. of the output & Related contexts\\
    \hline
    {\tt a} & $c_{M[{\cal C}[0]]}(\mathsf{ESC})c_{M[{\cal C}[-1]]}(\mathtt{a})$ & $1\cdot 1/(r+1)$ & $M[{\cal C}[0]]=\{\mathsf{ESC},2~\mathtt{a},1\}$\\
    {\tt b} & $c_{M[\texttt{a}]}(\mathsf{ESC})c_{M[{\cal C}[0]]}(\mathsf{ESC})c_{M[{\cal C}[-1]]}(\mathtt{b})$ & $1\cdot 2/3\cdot 1/r$ & $M[\mathtt{a}]=\{\mathsf{ESC},1~\mathtt{b},1\}$\\
    ~       & & & $M[c[0]]=\{\mathsf{ESC},3~\mathtt{a},1~\mathtt{b},1\}$\\
    {\tt a} & $c_{M[\texttt{b}]}(\mathsf{ESC})c_{M[{\cal C}[0]]}(\mathtt a)$ & $1\cdot 1/3$ & $M[\mathtt{b}]=\{\mathsf{ESC},1~\mathtt{a},1\}$\\
    ~       & & & $M[{\cal C}[0]]=\{\mathsf{ESC},3~\mathtt{a},2~\mathtt{b},1\}$\\
    {\tt b} & $c_{M[\mathtt{a}]}(\mathtt b)$ & $1/2$ & $M[{\cal C}[\mathtt{a}]]=\{\mathsf{ESC},1~\mathtt{b},2\}$\\
    {\tt c} & $c_{M[\mathtt{b}]}(\mathsf{ESC})c_{M[{\cal C}[0]]}(\mathsf{ESC})c_{M[{\cal C}[-1]]}(\mathtt c)$ & $1/2\cdot 1/4\cdot 1/(r-1)$ & $M[\mathtt{b}]=\{\mathsf{ESC},1~\mathtt{a},1~\mathtt{c},1\}$\\
    ~       & & & $M[{\cal C}[0]]=\{\mathsf{ESC},4~\mathtt{a},2~\mathtt{b},1~\mathtt{c},1\}$\\
    {\tt b} & $c_{M[\texttt{c}]}(\mathsf{ESC})c_{M[{\cal C}[0]]}(\mathtt b)$ & $1\cdot 1/5$ & $M[\mathtt{c}]=\{\mathsf{ESC},1~\mathtt{b},1\}$\\
    ~       & & & $M[{\cal C}[0]]=\{\mathsf{ESC},4~\mathtt{a},2~\mathtt{b},2~\mathtt{c},1\}$\\ 
    {\tt a} & $c_{M[\mathtt{b]}}(\mathtt a)$ & $1/3$ & $M[\mathtt{b}]=\{\mathsf{ESC},1~\mathtt{a},2~\mathtt{c},1\}$\\
    {\tt b} & $c_{M[\mathtt{a}]}(\mathtt b)$ & $2/3$ & $M[\mathtt{a}]=\{\mathsf{ESC},1~\mathtt{b},3\}$\\
    {\tt a} & $c_{M[\mathtt{b}]}(\mathtt a)$ & $2/4$ & $M[\mathtt{b}]=\{\mathsf{ESC},1~\mathtt{a},3~\mathtt{c},1\}$\\
    {\tt b} & $c_{M[\mathtt{a}]}(\mathtt b)$ & $3/4$ & $M[\mathtt{a}]=\{\mathsf{ESC},1~\mathtt{b},4\}$\\
    {\tt a} & $c_{M[\mathtt{b}]}(\mathtt a)$ & $3/5$ & $M[\mathtt{b}]=\{\mathsf{ESC},1~\mathtt{a},4~\mathtt{c},1\}$\\
    {\tt a} & $c_{M[\mathtt{a}]}(\mathsf{ESC})c_{M[{\cal C}[0]]}(\mathtt a)$ & $1/5\cdot 2/4$ & $M[\mathtt{a}]=\{\mathsf{ESC},1~\mathtt{a},1~\mathtt{b},4\}$\\
    ~       & & & $M[{\cal C}[0]]=\{\mathsf{ESC},4~\mathtt{a},3~\mathtt{b},2~\mathtt{c},1\}$\\
    {\tt a} & $c_{M[\mathtt{a}]}(\mathtt a)$ & $1/6$ & $M[\mathtt{a}]=\{\mathsf{ESC},1~\mathtt{a},2~\mathtt{b},4\}$\\
    {\tt a} & $c_{M[\mathtt{a}]}(\mathtt a)$ & $2/7$ & $M[\mathtt{a}]=\{\mathsf{ESC},1~\mathtt{a},3~\mathtt{b},4\}$\\
    {\tt a} & $c_{M[\mathtt{a}]}(\mathtt a)$ & $3/8$ & $M[\mathtt{a}]=\{\mathsf{ESC},1~\mathtt{a},4~\mathtt{b},4\}$\\
    {\tt a} & $c_{M[\mathtt{a}]}(\mathtt a)$ & $4/9$ & $M[\mathtt{a}]=\{\mathsf{ESC},1~\mathtt{a},5~\mathtt{b},4\}$\\
    {\tt a} & $c_{M[\mathtt{a}]}(\mathtt a)$ & $5/10$ & $M[\mathtt{a}]=\{\mathsf{ESC},1~\mathtt{a},6~\mathtt{b},4\}$
  \end{tabular}
\end{center}
\caption{An example of context-based statistical
  encoding.\label{fig:PPM}}
\end{figure}
\end{comment}
\begin{figure}
  \svgfig{graphics/PPM_example}{8cm}{800} %
  \caption{An example of context-based statistical
  encoding.} %
  \label{fig:PPM}
\end{figure}


\section{Decoder}
\begin{enumerate}
\item Equal to the step 1 of the encoder.
\item While the input is not exhausted:
  \begin{enumerate}
  \item $i\leftarrow k$ (except for the first symbol, where
    $i\leftarrow 0$).
  \item $s\leftarrow$ next decoded symbol.
  \item While $s=\mathtt{ESC}$:
    \begin{enumerate}
      \item Update $p(\mathtt{ESC}|{\cal C}[i])$.
      \item $i\leftarrow i-1$.
      \item $s\leftarrow$ next decoded symbol.
    \end{enumerate}
  \item Update $p(s|{\cal C}[i])$.
  \item While $i<k$:
    \begin{enumerate}
    \item $i\leftarrow i+1$.
    \item Update $p(s|{\cal C}[i])$ (insert $s$ into the ${\cal C}[i]$
      context).
    \end{enumerate}
  \end{enumerate}
\end{enumerate}

% \subsection*{Ejemplo}
% \begin{center}
%   \resizebox{0.85\textwidth}{!}{
%     \begin{tabular}{rrrl}
%      Entrada &Bits & Salida & Contextos afectados\\
%      \hline
%      $c_{\mathtt a}[-1]$            & $8.0056$ & {\tt a} & $c[\emptyset]=\{\mathsf{ESC},3~\mathtt{a},1\}$\\
%      $c_{\mathsf{ESC}}[\emptyset]$  & $1.5849$ &         & $c[\emptyset]=\{\mathsf{ESC},3~\mathtt{a},1\}$\\
%      $c_{\mathtt b}[-1]$            & $8     $ & {\tt b} & $c[\mathtt{a}]=\{\mathsf{ESC},1~\mathtt{b},1\}, c[\emptyset]=\{\mathsf{ESC},4~\mathtt{a},1~\mathtt{b},1\}$\\
%      $c_{\mathtt a}[\emptyset]$     & $2     $ & {\tt a} & $c[\mathtt{b}]=\{\mathsf{ESC},1~\mathtt{a},1\}, c[\emptyset]=\{\mathsf{ESC},4~\mathtt{a},2~\mathtt{b},1\}$\\
%      $c_{\mathtt b}[\mathtt{a}]$    & $1     $ & {\tt b} & $c[\mathtt{a}]=\{\mathsf{ESC},1~\mathtt{b},2\}$\\
%      $c_{\mathsf{ESC}}[\mathtt{b}]$ & $1     $ &         & $c[\mathtt{b}]=\{\mathsf{ESC},2~\mathtt{a},1\}$\\
%      $c_{\mathsf{ESC}}[\emptyset]$  & $0.58  $ &         & $c[\emptyset]=\{\mathsf{ESC},5~\mathtt{a},2~\mathtt{b},1\}$\\
%      $c_{\mathtt c}[-1]$            & $7.98  $ & {\tt c} & $c[\mathtt{b}]=\{\mathsf{ESC},2~\mathtt{a},1~\mathtt{c},1\}, c[\emptyset]=\{\mathsf{ESC},5~\mathtt{a},2~\mathtt{b},1~\mathtt{c},1\}$\\
%      $c_{\mathtt b}[\emptyset]$     & $2.81  $ & {\tt b} & $c[\mathtt{c}]=\{\mathsf{ESC},1~\mathtt{b},1\}, c[\emptyset]=\{\mathsf{ESC},5~\mathtt{a},2~\mathtt{b},2~\mathtt{c},1\}$\\ 
%      $c_{\mathtt a}[\mathtt{b}]$    & $2     $ & {\tt a} & $c[\mathtt{b}]=\{\mathsf{ESC},2~\mathtt{a},2~\mathtt{c},1\}$\\
%      $c_{\mathtt b}[\mathtt{a}]$    & $0.58  $ & {\tt b} & $c[\mathtt{a}]=\{\mathsf{ESC},1~\mathtt{b},3\}$\\
%      $c_{\mathtt a}[\mathtt{b}]$    & $1.32  $ & {\tt a} & $c[\mathtt{b}]=\{\mathsf{ESC},2~\mathtt{a},3~\mathtt{c},1\}$\\
%      $c_{\mathtt b}[\mathtt{a}]$    & $0.42  $ & {\tt b} & $c[\mathtt{a}]=\{\mathsf{ESC},1~\mathtt{b},4\}$\\
%      $c_{\mathtt a}[\mathtt{b}]$    & $1     $ & {\tt a} & $c[\mathtt{b}]=\{\mathsf{ESC},2~\mathtt{a},4~\mathtt{c},1\}$\\
%      $c_{\mathsf{ESC}}[\mathtt{a}]$ & $2.32  $ &         & $c[\mathtt{a}]=\{\mathsf{ESC},2~\mathtt{b},4\}$\\
%      $c_{\mathtt a}[\emptyset]$     & $1.58  $ & {\tt a} & $c[\mathtt{a}]=\{\mathsf{ESC},2~\mathtt{b},4~\mathtt{a},1\}, c[\emptyset]=\{\mathsf{ESC},5~\mathtt{a},3~\mathtt{b},2~\mathtt{c},1\}$\\
%      $c_{\mathtt a}[\mathtt{a}]$    & $2.81  $ & {\tt a} & $c[\mathtt{a}]=\{\mathsf{ESC},2~\mathtt{b},4~\mathtt{a},2\}$\\
%      $c_{\mathtt a}[\mathtt{a}]$    & $2     $ & {\tt a} & $c[\mathtt{a}]=\{\mathsf{ESC},2~\mathtt{b},4~\mathtt{a},3\}$\\
%      $c_{\mathtt a}[\mathtt{a}]$    & $1.58  $ & {\tt a} & $c[\mathtt{a}]=\{\mathsf{ESC},2~\mathtt{b},4~\mathtt{a},4\}$\\
%      $c_{\mathtt a}[\mathtt{a}]$    & $1.32  $ & {\tt a} & $c[\mathtt{a}]=\{\mathsf{ESC},2~\mathtt{b},4~\mathtt{a},5\}$\\
%      $c_{\mathtt a}[\mathtt{a}]$    & $1.14  $ & {\tt a} & $c[\mathtt{a}]=\{\mathsf{ESC},2~\mathtt{b},4~\mathtt{a},6\}$
%    \end{tabular}
%  }
%\end{center}
% }}} }}}

\section*{Let's go the the lab!}
\begin{enumerate}
\item
\begin{verbatim}
         Codec | lena boats pepers zelda Average
---------------+--------------------------------
             : |    :     :      :     :       :
arith-n-c -o 1 | ....  ....   ....  ....    ....
arith-n-c -o 2 | ....  ....   ....  ....    ....
arith-n-c -o 3 | ....  ....   ....  ....    ....
\end{verbatim}
\item Check the reversibility.
\end{enumerate}

\bibliography{text-compression}
